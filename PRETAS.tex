\textbf{Danilo Paiva Ramos} é antropólogo e professor adjunto do Departamento de Antropologia e Etnologia da Universidade Federal da Bahia \textsc{(ufba)}. Tem mestrado, doutorado e pós-doutorado em Antropologia Social pela Universidade de São Paulo \textsc{(usp)}, e pós-doutorado em Antropologia Linguística pela Universidade do Texas \textsc{(ut)}. Além de suas pesquisas etnográficas, com ênfase em xamanismo, discurso, performance, vida ritual e territorialidade, é engajado em causas ligadas aos direitos indígenas. Paralelo à coordenação do Grupo de Estudos de Antropologia e Linguística (\textsc{geal}) da \textsc{usp}, participa do Coletivo de Apoio à Causa Yuhup-Hup (\textsc{capyh}), é assessor da Federação da Organizações Indígenas do Rio Negro (\textsc{foirn}) em saúde indígena e da Funai em assuntos e projetos voltados aos povos Hupd'äh e Yuhupdëh. Também assessora o povo Hupd'äh para a construção de seus Planos de Gestão Territorial e Ambiental (\textsc{pgta--hup}).

\textbf{Círculos de coca e fumaça} debruça-se sobre os Hupd’äh, povo indígena falante de língua hup que vive na região do Alto Rio Negro, no noroeste da Amazônia. Suas rodas noturnas para ingerir coca e tabaco -- momentos de compartilhar mitos e histórias de andanças pela mata, ensinar benzimentos e executar curas e proteções xamânicas -- são o principal cenário do livro. Nessas situações, Paiva Ramos percebeu performances, contextos em que os ameríndios relacionam suas experiências e observações da mata com as palavras dos mitos e encantamentos. A partir dessa interação, o viajante hup consegue interagir com seres de múltiplas paisagens e expandir seu campo de percepção, em um engajamento mútuo com os processos de transformação do mundo.

\textbf{Coleção Mundo Indígena} reúne materiais produzidos com pensadores de diferentes povos indígenas e pessoas que pesquisam, trabalham ou lutam pela garantia de seus direitos. Os livros foram feitos para serem utilizados pelas comunidades envolvidas na sua produção, e por isso uma parte significativa das obras é bilíngue. Esperamos divulgar a imensa diversidade linguística dos povos indígenas no Brasil, que compreende mais de 150 línguas pertencentes a mais de trinta famílias linguísticas.

