\textbf{Danilo Paiva Ramos} é antropólogo, professor adjunto do Departamento de Ciências Humanas da Universidade Federal de Alfenas (\textsc{unifal--mg}) e membro efetivo do Programa de Pós-graduação em Antropologia da Universidade Federal da Bahia (\textsc{ppga--ufba}). Tem pós-doutorado em Antropologia Linguística pela Universidade de São Paulo (\textsc{usp}), Universidade do Texas (\textsc{ut}), doutorado em Antropologia Social --- Etnologia indígena --- pela (\textsc{usp}), mestrado em Antropologia Social --- Antropologia Rural --- também pela (\textsc{usp}), graduação em Ciências Sociais (\textsc{usp}) e licenciatura em Ciências Sociais (\textsc{usp}). É coordenador do Grupo de Pesquisa em Etnologia, Linguística e Saúde Indígena (\textsc{etnolinsi}), pelo \textsc{cnp}q. Desenvolve pesquisas em etnologia indígena, com ênfase em estudos sobre xamanismo, línguas indígenas, arte verbal e saúde indígena. É membro da Associação Brasileira de Antropologia (\textsc{aba}) e do Centro de Estudos Ameríndios (\textsc{ce}st\textsc{a}--\textsc{usp}). Integra o Coletivo de Apoio aos Povos Yuhupdëh, Hupd'äh, Dâw e Nadëb (\textsc{capyhdn}) e a Associação Saúde Sem Limites (\textsc{ssl}). É assessor do povo Hupd'äh para a implementação de seu Plano de Gestão Territorial e Ambiental (\textsc{pgta}--Hup).

\textbf{Círculos de coca e fumaça} é um ensaio sobre os Hupd’äh, povo indígena falante de língua hup que vive na região do Alto Rio Negro, no noroeste da Amazônia. Suas rodas noturnas destinadas à ingestão de coca e tabaco --- momento de compartilhamento de mitos e histórias de andanças pela mata, ensinamento de benzimentos e execução de curas e proteções xamânicas --- são o principal cenário desse livro. Nessas situações são percebidas performances, contextos em que os ameríndios relacionam suas experiências e observações da mata com as palavras dos mitos e encantamentos. A partir dessa interação, o viajante hup consegue interagir com seres de múltiplas paisagens e expandir seu campo de percepção, em um engajamento mútuo com os processos de transformação do mundo.

\textbf{Mundo Indígena} reúne materiais produzidos com pensadores de diferentes povos indígenas e pessoas que pesquisam, trabalham ou lutam pela garantia de seus direitos. Os livros foram feitos para serem utilizados pelas comunidades envolvidas na sua produção, e por isso uma parte significativa das obras é bilíngue. Esperamos divulgar a imensa diversidade linguística dos povos indígenas no Brasil, que compreende mais de 150 línguas pertencentes a mais de trinta famílias linguísticas.


