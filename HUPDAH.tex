\chapter{Para ler as palavras hup}

Para a grafia dos termos da língua hup em geral, adotou"-se como
referência o dicionário de língua hup elaborado pelo linguista Henri
Ramirez, \textit{A língua dos Hupd'äh do Alto Rio Negro}.\footnote{Associação Saúde
Sem Limites, de 2006.} Todos os termos em língua hup são colocados entre
barras e seguidos ou precedidos pela tradução entre aspas. Por exemplo: 
\textit{Ta̗t"-Dëh}, ``taracuá"-igarapé''. Seguindo Ramirez, mantenho a acentuação
das vogais de acordo com a nasalidade --- indicada por um \textit{til} --- e o tom --- indicado por um acento grave agudo ou grave.

\section{o alfabeto}

Ramirez propõe que o alfabeto hup tem 25 letras: \textit{a}, \textit{ä}, \textit{b}, \textit{ç}, \textit{e}, \textit{ë}, \textit{g},
\textit{h}, \textit{i}, \textit{ɨ}, \textit{j}, \textit{k}, \textit{m}, \textit{n}, \textit{o}, \textit{ö}, \textit{p}, \textit{r}, \textit{s}, \textit{t}, \textit{u}, \textit{w}, y e '.\footnote{Oclusão
glotal.} Destas, 16 são consoantes, nove são vogais e 11 são consoantes
laringalizadas: \textit{b'}, \textit{d'}, \textit{r'}, \textit{j'}, \textit{g'}, \textit{m'}, \textit{n'}, \textit{w'}, \textit{y'}, \textit{s'}, \textit{k'}.

\section{genealogia}

Ao longo dos capítulos, na primeira vez em que é feita referência ao nome de
uma pessoa, este é seguido por seu nome em língua hup, por sua data de
nascimento, por seu clã e por seu número de indivíduo na \textit{base de dados
populacionais}.\footnote{Como no exemplo de Ponciano, dentro do texto, que traz uma frase complementar após seu nome. No caso, \textit{Hu̖d}, 5 de julho de 1946, \textit{Sokw'ä̗t Noh K'öd Tẽ̖h}.} 

Adota"-se também a notação inglesa para as referências às posições genealógicas:

\begingroup
\begin{tabular}{l}
\textsc{f}, \textit{pai}\\
\textsc{m}, \textit{mãe}\\
\textsc{b}, \textit{irmão}\\
\textsc{z}, \textit{irmã}\\
\textsc{h}, \textit{marido}\\
\textsc{w}, \textit{esposa}\\
\textsc{s}, \textit{filho}\\
\textsc{d}, \textit{filha}\\
\textsc{e}, \textit{mais velho\,(a)}\\
\textsc{y}, \textit{mais novo\,(a)}\\
\end{tabular}

%\section{genealogia}
As narrativas de mitos e sonhos e as exegeses de benzimentos são
destacadas do texto pelas letras \textsc{m}, como \textit{mito}, \textsc{s}, como \textit{sonho}, e
\textsc{b}, \textit{benzimento}. Todas são numeradas sequencialmente, seguidas
pelo título e pela diagramação específica. Ao longo da análise, as
referências a esses textos são feitas através do uso das mesmas letras.\footnote{Por exemplo, \textsc{m1}.} Para algumas narrativas míticas curtas, optou"-se
por destacá"-las apenas pela letra \textsc{m}, numerada e sem negrito. Ao fim do
livro podem ser encontrados índices dos mitos e benzimentos.


% \begin{table}[ht!]
% \parbox{.45\linewidth}{
% \centering
% \begin{tabular}{ccccc}
% \hline
% \multicolumn{5}{c}{\textsc{consoantes}}                                  \\ \hline
% P & t                                               & s & k & ’ \\ \hline
% B & \begin{tabular}[c]{@{}c@{}}d\\ (r)\end{tabular} & j & g &   \\ \hline
% M & n                                               &   &   &   \\ \hline
%   &                                                 & ç &   & h \\ \hline
% W &                                                 & y &   &   \\ \hline
% \end{tabular}
% }
% \hfill
% \parbox{.45\linewidth}{
% \centering
% \begin{tabular}{ccc}
% \hline
% \multicolumn{3}{c}{\textsc{vogais}} \\ \hline
% i       & ɨ       & U      \\ \hline
% ë       & Ä       & ö      \\ \hline
% e       & A       & o      \\ \hline
% \end{tabular}
% }
% \end{table}

% \bigskip