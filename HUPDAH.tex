\chapter{Para ler as palavras hup}

Para a grafia dos termos da língua hup em geral, adotou"-se como
referência o dicionário de língua hup elaborado pelo linguista Henri
Ramirez, \emph{A língua dos Hupd'äh do Alto Rio Negro} (Associação Saúde
Sem Limites, 2006). Todos os termos em língua hup são colocados entre
barras e seguidos ou precedidos pela tradução entre aspas (ex.
\textit{Ta̗t"-Dëh}, ``taracuá"-igarapé''). Seguindo Ramirez, mantenho a acentuação
das vogais de acordo com a nasalidade (indicada por um til) e o tom
(indicado por um acento grave agudo ou grave).

Ramirez propõe que o alfabeto hup possui 25 letras: a, ä, b, ç, e, ë, g,
h, i, ɨ, j, k, m, n, o, ö, p, r, s, t, u, w, y, ' (oclusão
glotal). Destas, 16 são consoantes, 9 são vogais e 11 são consoantes
laringalizadas (b', d', r', j', g', m', n', w', y', s', k').

\bigskip

\begin{table}[ht!]
\parbox{.45\linewidth}{
\centering
\begin{tabular}{ccccc}
\hline
\multicolumn{5}{c}{\textsc{consoantes}}                                  \\ \hline
P & t                                               & s & k & ’ \\ \hline
B & \begin{tabular}[c]{@{}c@{}}d\\ (r)\end{tabular} & j & g &   \\ \hline
M & n                                               &   &   &   \\ \hline
  &                                                 & ç &   & h \\ \hline
W &                                                 & y &   &   \\ \hline
\end{tabular}
}
\hfill
\parbox{.45\linewidth}{
\centering
\begin{tabular}{ccc}
\hline
\multicolumn{3}{c}{\textsc{vogais}} \\ \hline
i       & ɨ       & U      \\ \hline
ë       & Ä       & ö      \\ \hline
e       & A       & o      \\ \hline
\end{tabular}
}
\end{table}

\bigskip

Adota"-se a notação inglesa para as referências às posições genealógicas:

\begin{quote}
\textbf{F} = pai, \textbf{M} = mãe, \textbf{B} = irmão, \textbf{Z} =
irmã, \textbf{H} = marido, \textbf{W} = esposa, \textbf{S} = filho,
\textbf{D} = filha, \textbf{e} = mais velho (a), \textbf{y} = mais novo
(a).
\end{quote}

\medskip

Ao longo dos capítulos, na primeira vez que é feita referência ao nome de
uma pessoa, este é seguido por seu nome em língua hup, por sua data de
nascimento, por seu clã e por seu número de indivíduo na ``Base de dados
populacionais'' em anexo (Anexo 2) como no exemplo a seguir:

Ponciano (\textit{Hu̖d}, 05/07/1946, \textit{Sokw'ä̗t Noh K'öd Tẽ̖h,} ind.10)

As narrativas de mitos e sonhos e as exegeses de benzimentos são
destacadas do texto pelas letras \textsc{m} (mito), \textsc{s} (sonho) e
\textsc{b} (benzimento) numeradas sequencialmente, seguidas
pelo título e pela diagramação específica. Ao longo da análise, as
referências a esses textos são feitas através do uso das mesmas letras
(ex. \textsc{m1}). Para algumas narrativas míticas curtas, optou"-se
por destacá"-las apenas pela letra \textsc{m} numerada e sem negrito. Ao fim do
livro podem ser encontrados índices dos mitos e benzimentos.