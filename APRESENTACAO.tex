%\chapter[Apresentação]{Apresentação\subtitulo{Coca, tabaco e caxiri}}
\chapter*{Apresentação\smallskip\subtitulo{Coca, tabaco e caxiri}}
\addcontentsline{toc}{chapter}{Apresentação, \textit{por Sylvia Caiuby Novaes}}


\begin{flushright}
\textsc{sylvia caiuby novaes}
\end{flushright}

\noindent{}No final da tarde, de qualquer casa desta aldeia hupd'äh à beira do
Taracuá"-Igarapé, às margens do rio Tiquié, é possível ouvir o som
ritmado do pilão, onde as folhas de coca são misturadas às folhas
queimadas de imbaúba. As rodas de coca reúnem todas as noites cerca de
10 homens que, sentados em círculos, comem a coca, fumam tabaco e bebem
caxiri. É a partir da participação de Danilo nestas rodas e da
etnografia destes três elementos --- a coca, o tabaco e o caxiri --- que
somos levados a conhecer o mundo hupd'äh, este povo maku, habitante do
Alto Rio Negro, na fronteira entre o Brasil e a Colômbia. Analisadas por
Danilo como \textit{performances}, as rodas de coca nos introduzem ao
xamanismo e aos \textit{benzimentos}, fundamentais nas atividades de cura.

Danilo observa que nestas rodas de coca as interações delineiam ``os
caminhos e viagens xamânicas como percursos de observação e ação''.
Viagens e narrativas de viagens são um verdadeiro fascínio para os
Hupd'äh. Tal como os ancestrais dos diversos clãs hup que empreendiam
viagens na Cobra"-Canoa, este livro de Danilo nos transporta para
paisagens no Alto Rio Negro com um encantamento contagiante. Danilo
acompanha os Hupd'äh em suas viagens pelos caminhos que atravessam a
floresta, viagens para pesca, para outras aldeias, para as terras dos
antepassados. Nessas viagens são constantes as interações com animais,
plantas, ancestrais e vários outros seres. Ao caminhar e viajar, assim
como nos deslocamentos da pessoa ao benzer ou sonhar, os Hupd'äh agem no
sentido da transformação de pessoas e de perspectivas, processo que
Danilo analisa conjugando as ricas abordagens de Claude Lévi"-Strauss,
Victor Turner, Tim Ingold e Peter Gow. Tendo como quadro teórico Carlo
Severi, Richard Schechner, Richard Bauman e Laura Graham, a memória
ritual é retomada pelo autor e analisada como engajamento perceptual com
o ambiente.

A partir de uma rigorosa revisão da literatura sobre os povos do alto
rio Negro, de Koch"-Grünberg no início do século \textsc{xx} a antropólogos
contemporâneos que vêm trabalhando na região, como Pedro Lolli, Danilo
procura entender as direções percorridas por este povo indígena ao longo
de suas existências. Além de muitas histórias que ouviu ao longo de suas
viagens com os Hupd'äh, Danilo registrava com a máquina fotográfica e o
\textsc{gps} os caminhos percorridos. As fotos, as linhas e pontos do \textsc{gps} e seu
caderno ordenavam sua experiência de campo e permitiam rememorar com os
Hupd'äh o que haviam percorrido.

À análise da coca Danilo contrapõe a análise do tabaco e da fumaça que
ele gera, e é ao se sentar em roda para comer coca e fumar que se forma
o Lago de Leite, ``poderosa paisagem de vida que estabelece a presença
imanente do espaço"-tempo da criação, da dádiva e da possibilidade de
cura, proteção e regeneração''. Lago de Leite que também se forma no
ventre e peitos maternos ao longo da gravidez, nos mostra Danilo, quando
discorre sobre partos, nascimentos e corporalidade até chegar aos
instrumentos, estes também corpos.

O caxiri, preparado pelas mulheres, é o terceiro dos elementos
analisados por Danilo. Há uma socialidade entre as manivas: são
plantadas próximas uma às outras, formando rodas de conversa para que
cresçam, tal como ocorre com as plantações de coca, plantadas ao lado
das manivas. ``O caxiri e a coca são veículos de interação social que
agregam pessoas, plantas e humanos, em torno do consumo das substâncias,
das conversas e dos encantamentos''.

Este livro é o resultado de uma pesquisa de doutorado que tive o
privilégio de orientar. Danilo Paiva Ramos é um antropólogo que investiu
na longa permanência em campo, que aproximou"-se da linguística como
campo fértil para a antropologia, que conseguiu fluência no idioma
hupd'äh, e mais, tem sempre em mente o compromisso político com seus
interlocutores. Sua proximidade com os Hupd'äh permitiu"-lhe compartilhar
com este povo caminhos, palavras e paisagens, que ele agora nos oferece
neste belo trabalho. Sua etnografia sofisticada tem como resultado a
possibilidade de partir de um universo circunscrito --- as rodas de coca
--- e chegar à interação cosmológica entre animais, plantas, espíritos e
à vida, que entre os Hupd'äh é criada e recriada em processos que
cruzam, como num emaranhado, a caça, os nascimentos e os benzimentos. O
texto de Danilo evidencia a continuidade entre as posturas e os gestos,
que o pesquisador capta com olhar atento, apresentando"-nos este
\textit{continuum} entre o corpo, a paisagem e os objetos.

Danilo tem uma construção textual que produz, ela própria,
encantamentos. Ao aliar a escuta à escrita etnográfica, com fina
sensibilidade, Danilo nos proporciona uma entrada única no mundo
hupd'äh. Trata"-se do segundo livro deste jovem antropólogo e já se pode
perceber um estilo em sua narrativa. Tal como em \textit{Nervos da terra}
(2009), a sensibilidade da escrita vem de sua sólida formação
teórica nos vários campos da antropologia e da proximidade com seus
interlocutores, que se tornam parceiros e permitem uma abordagem
original e ao mesmo tempo fascinante.